\documentclass[11pt]{article}
\usepackage{graphicx}
\usepackage{latexsym}
\usepackage{times}
\usepackage{amsmath}
\newtheorem{theorem}{Theorem}
\newtheorem{definition}{Definition}
\newtheorem{corollary}{Corollary}
\newtheorem{strategy}{Strategy}

\newcommand{\ra}{\rightarrow}
\newcommand{\tagg}[1]{\noindent \textbf{{#1}} \\ }
\newcommand{\floor}[1]{\lfloor {#1} \rfloor}
\newcommand{\ceil}[1]{\lceil {#1} \rceil}

\setlength{\topmargin}{0in}
\setlength{\textwidth}{6.5in}
\setlength{\oddsidemargin}{0in}
\setlength{\evensidemargin}{0in}

\thispagestyle{empty}
%\pagestyle{empty}

%================================================================

\begin{document}

\begin{center}
CS173: Intermediate Programming \\
\Large{Complex ADT}
\end{center}

\begin{tabular}{|p{1in}|p{4in}|}
\hline
\multicolumn{2}{|l|}{Overview} \\ \hline
\multicolumn{2}{|p{5in}|}{
The Complex data structure stores and manipulates complex numbers of the form $a+bi$ where $a$ and $b$ are real numbers.  This Complex type supports a multitude of arithmetic operations pertinent to complex numbers; these operations are described below.
}  \\
\hline
\hline
\multicolumn{2}{|l|}{Constructors} \\ \hline
default & A new Complex data type should default to the value of $0+0i$.  \\
 & \texttt{Complex c; } \\
\hline
copy & Create a new Complex type from an existing one. \\
& \texttt{Complex c1(c2); } \\
\hline
Complex(a,b) & We should be able to specify a new Complex type by giving its real and imaginary components.   The imaginary part defaults to 0.  \\
& \texttt{Complex c1(2,3.1);  Complex c2(2.5);} \\
\hline
\hline
\multicolumn{2}{|l|}{Operators} \\ \hline
addition & Should support addition between two complex numbers, complex and int, and complex and float.  \\
& \texttt{c1 = c2 + c3;  c1 = c2 + 5.5; } \\
\hline
subtraction & Should support subtraction between two complex numbers, complex and int, and complex and float.  \\
& \texttt{c1 = c2 - c3;  c1 = c2 - 5; } \\
\hline
multiplication & Should support multiplication between two complex numbers, complex and int, and complex and float.  \\
& \texttt{c1 = c2 * c3;  c1 = c2 * 5.5; } \\
\hline
division & Should support division between two complex numbers, complex and int, and complex and float.  \\
& \texttt{c1 = c2 / c3;  c1 = c2 / 5; } \\
\hline
conjugate & The $\sim$ operator returns the complex conjugate.  \\
& \texttt{c1 = $\sim$c2;} \\
\hline
negation & The $-$ operator returns the negative of a complex.  \\
& \texttt{c1 = -c2;} \\
\hline
exponentiation & The $\wedge$ operator should raise a complex number to an integer power. \\
& \texttt{c1 = c2$\wedge$x;} (where $x$ is integer only)\\
\hline
abs & The abs method should return the distance from the origin. \\
& \texttt{c1.abs()}  \\
\hline
\end{tabular}

\begin{tabular}{|p{1in}|p{4in}|}
\hline
\multicolumn{2}{|l|}{Modifiers and Accessors} \\ \hline
setReal & Sets the real part of the complex number. \\
   & \texttt{c1.setReal(5);} \\
\hline
getReal & Gets the real part of the complex number. \\
   & \texttt{float f = c1.getReal();} \\
\hline
setImag & Sets the imaginary part of the complex number. \\
   & \texttt{c1.setImag(5);} \\
\hline
getImag & Gets the imaginary part of the complex number. \\
   & \texttt{float f = c1.getImag();} \\
\hline
\hline
\multicolumn{2}{|l|}{Other} \\ \hline
assignment & Allows assignment of values between complex numbers. \\
operator & \texttt{c1 = c2;} \\
\hline
destructor & Cleans up the complex class. \\
\hline
equality & Equal if both real and imaginary parts are equal. \\
operator & \texttt{c1 == c2;}  \\
\hline
inequality &True if either real or imaginary parts are not equal. \\
operator & \texttt{c1 != c2;} \\
\hline
greaterThan & Returns true if abs(c1) \texttt{>} abs(c2), false otherwise. \\
operator & \texttt{c1 > c2;} \\
\hline
greaterEqual & Returns true if abs(c1) \texttt{>=} abs(c2), false otherwise. \\
operator & \texttt{c1 >= c2;} \\
\hline
lessThan & Returns true if abs(c1) \texttt{<} abs(c2), false otherwise. \\
operator & \texttt{c1 < c2;} \\
\hline
lessEqual & Returns true if abs(c1) \texttt{<=} abs(c2), false otherwise. \\
operator & \texttt{c1 <= c2;} \\
\hline
\texttt{cout <<} & Allows printing of a complex number as a string "$a+bi$". \\
& \texttt{ cout << c1 << endl;} \\
& print \texttt{a+0i} as a \\
& print \texttt{0+bi} as bi \\
& print \texttt{a+-bi} as a-bi \\
\hline
\texttt{cin>>} & Allows reading of a complex number as a string "$a$+$bi$". \\
& \texttt{ cin >> c1;} \\
& reads: \texttt{a+bi, a-bi, -a+bi, -a-bi, a+-bi, +a+bi} \\
& reads: \texttt{a, -a, +a, bi, -bi, +bi} \\
& where \texttt{a,b} can be integers or reals with decimal points \\
\hline
\end{tabular}


\end{document}


 